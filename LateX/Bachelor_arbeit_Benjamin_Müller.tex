\documentclass[a4paper,12pt,oneside, bibliography=totoc]
{scrbook}
\usepackage[utf8]{inputenc}

% Schrift und -kodierung
\usepackage[T1]{fontenc}
\usepackage{lmodern}

% Sprache/Silbentrennung
\usepackage[german]{babel} 
\usepackage{booktabs}
\usepackage{amsmath}
\usepackage{floatflt}
\usepackage{float} 
\usepackage{graphicx}
\usepackage{pbox}
\usepackage{algorithmic}
\usepackage{algorithm}
\usepackage{siunitx}


\usepackage[page, title, titletoc, header]{appendix} %prettier appendix



\usepackage[printonlyused]{acronym}
\usepackage{csquotes}
\usepackage{listings} 
\usepackage{subfig}
\lstset{xleftmargin=2em} %Proper indention of listings


\usepackage{tabularx} %For tables
\usepackage{csquotes} %For Quotes





%Footnote Numbering not reset in new chapters
\usepackage{chngcntr}
\counterwithout{footnote}{chapter}


%Remove last point after section/subsections
\renewcommand{\autodot}{}

\usepackage[htt]{hyphenat} %damit texttt noch Linebreaks mit Silbentrennung erzeugt
\newcommand{\code}[1]{\texttt{#1}} %Programmcode im Textfluss in passendem Font ausgeben




   
%Literatur
%Ordering in references checken, vermutlich was mit style=numeric zu tun
\usepackage[
   backend=biber,
   sorting= none,
   firstinits=true,
   date=long,
   urldate=long
]{biblatex}
\addbibresource{literatur1.bib}
\addbibresource{literatur2.bib}


\usepackage[]{hyperref}

\begin{document}
\frontmatter %roman page numbers





	\titlehead{
	\begin{center}
	   \includegraphics[width=10cm]{figures/unilogo.pdf}\\
	   	Institut für Informatik, AG Software Engineering
	\end{center}
	}
	\subject{Bachelorarbeit}
	\title{Entwicklung eines Schwachstellenscanners  für Python Webapplikationen}
	\author{Benjamin Müller \\ Immatrikulationsnummer: 962010} 

	\date{30.11.2020 \\
	Erstbetreuerin: Prof. Dr.-Ing. Elke Pulvermüller \\
	Zweitbetreuer: } 
	
	\maketitle
	
	\clearpage
	
	\addchap*{Abstract}
\textbf{Deutsch}
Sicherheitsstandards wie IEC-61508 oder ISO26262 stellen Richtlinien für die Entwicklung eingebetteter Systeme in sicherheitskritischen Bereichen bereit. Allerdings beinhalten sie keinerlei Unterstützung für die automatische Implementierung von Mechanismen der funktionalen Sicherheit. Das Ziel dieser Arbeit ist es solch eine automatische Implementierung mithilfe der modellgetriebenen Entwicklung zu ermöglichen. Um dieses Ziel zu erreichen werden ausgehend von den Sicherheitsempfehlungen des IEC-61508 Standards fehlererkennende und -korrigierende Codes genutzt um einen software-basierten Speicherschutz zu ermöglichen. Hierzu werden eine erweiterbare Software-Architektur und dazugehörige Modelltransformationen entwickelt. Diese werden anschließend im Rahmen einer prototypischen Implementierung experimentell evaluiert.



%\linebreak
\bigskip

\noindent
%\bigskip
\textbf{English} 
Standards such as IEC-61508 or ISO26262 provide a general guideline on how to develop embedded systems in safety-critical contexts. However, they offer no actual support for the implementation of safety mechanisms. The goal of this thesis is to provide such development support by employing Model Driven Development. Based on the safety recommendations of IEC-61508, Error Detecting and Correcting Codes are applied to implement software-based memory protection. For this, an extensible software architecture and corresponding model transformations are developed and evaluated experimentally.

%TODO Bis jetzt nur Osi Abstract, evtl. etwas ausführlicher für Masterarbeit


	\clearpage
	
	\tableofcontents
	\clearpage

	






\mainmatter %switch roman auf arabic page numbers
\chapter{Einleitung}
\setcounter{page}{1} %Seitenzahlen hier mit 1 anfangen

	%Include text from other files into the document --> great for structuring
	%\label{sec:introduction}
The introduction of the thesis.



Während dem Prozess der Softwareentwicklung nimmt die Software Sicherheit eine zunehmend wichtige Rolle ein. Unter ständig zunehmenden und sich diversifizierenden Cyberangriffen stehen Softwarehersteller vor der Herausforderung ihre Produkte gegen eine Vielzahl von Bedrohungen abzusichern und Hackern keine Angriffsflächen zu bieten. Dabei spielt die Vermeidung von Software Schwachstellen eine wichtige Rolle, da diese Einfallstore für böswillige Akteure darstellen. Insbesondere für Web Applikationen ist dies von hoher Priorität, da sie keine schützende Wand aus physischer und Netzwerksicherheit zwischen sich und der Außenwelt haben, sondern für jeden über HTTP Requests zugänglich sind.

Vielen Entwicklern fehlt detailliertes Wissen über Schwachstellen und ihre Vermeidung. Auch die Verwendung eines etablierten Web Frameworks schützt nur dann vor Schwachstellen, wenn der Entwickler es richtig einsetzt. Um dieses Problem zu bewältigen, bietet sich ein Schwachstellenscanner an, der den Quellcode überprüft und bei gefundenen Schwachstellen\footnote{Die Begriffe Schwachstelle und Sicherheitslücke werden in dieser Arbeit synonym benutzt. }  Lösungen vorshlägt. Der Konzeptionierung und Implementierung eins solchen Scanners widmet sich diese Arbeit. 

	
	\section{Anforderungen an den Scanner}
	\section{Methodik der Arbeit}
Im ersten Kapitel des Hauptteils werden bestimmte Schwachstellen nach ihrer Häufigkeit und ihrem Schweregrad\footnote{Mit Schweregrad ist hier der Grad der Ausnutzbarkeit und die technischen Folgen einer Sicherheitslücke gemeint}  ausgewählt
	\section{Relevante Arbeiten}



\chapter{Häufige Sicherheitslücken in Web Applikationen}
	
	\section{Injection}
	
	\section{Cross-Site Scripting}

	\section{Broken Authentification}

\chapter{Beschreibung der CRUD App}

	\section{Schwachstellen}

	
\chapter{Konzeption eines}
\chapter{Methodik des Scanners im Finden von Schwachstellen}

\chapter{Bewertung der Bibliothek}

\chapter{Zusammenfassung}



%Prints references
\printbibliography


\clearpage 


\chapter*{Erklärung zur selbstständigen Abfassung der Masterarbeit}

Ich versichere, dass ich die eingereichte Bachelorarbeit selbstständig und ohne unerlaubte Hilfe verfasst habe. Anderer als der von mir angegebenen Hilfsmittel und Schriften habe ich mich nicht bedient. Alle wörtlich oder sinngemäß den Schriften anderer Autoren entnommenen Stellen habe ich kenntlich gemacht. \\


\bigskip
\bigskip
\bigskip
\bigskip
\noindent
Osnabrück, 30. November 2020 \\

\bigskip
\bigskip
\bigskip
\bigskip
\noindent
Benjamin Müller
\end{document}

